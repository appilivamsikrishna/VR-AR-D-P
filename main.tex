\documentclass[10pt]{article}
\usepackage[utf8]{inputenc}
\usepackage[T1]{fontenc}
\usepackage{amsmath}
\usepackage{amsfonts}
\usepackage{amssymb}
\usepackage[version=4]{mhchem}
\usepackage{stmaryrd}

\title{VIRTUAL REALITY AND AUGMENTED REALITY IN DESIGN AND PRODUCTION }

\author{}
\date{}


\begin{document}
\maketitle
\begin{abstract}
Purpose - The purpose of the present study is to gather all the available information about how Virtual reality and Augmented reality are being used in designing, prototyping, and production processes. It specially focuses on how these technologies will be used more in the future for apparel production.
\end{abstract}

Through a comprehensive review of new technologies, the integration of virtual reality (VR) and augmented reality (AR) in the design and production processes of fashion products is explored. This article examines the potential of VR in areas such as virtual education, design, production, and manufacturing, highlighting the opportunities and challenges in using VR. In addition, it offers an advanced method for personalizing personal clothing, combined with interactive design with personal virtual images to complete the clothing design. The webbased 3D clothing customization system using Unity 3D and VR technology has been expanded with a focus on advanced techniques such as 3D face scanning, fabric selection and accessory selection. Experiments have shown how useful it is to give consumers authentic apparel and materials while offering a customized and engaging experience.

Furthermore, studies on the use of augmented reality (AR) in industrial systems to enhance training, productivity, and decision-making are still ongoing. Furthermore, a research paper illustrates how technologies like blockchain, Internet of Things (IOT), VR, AR, 3D printing, and artificial intelligence are integrated and how this affects the process of designing and building products. While artificial intelligence is emphasized as having a significant role in data analysis, predictive modelling, and design, VR and AR technologies are commended for offering a cohesive design environment. To personalize clothing in ways that an individual does not, this article explains the development of personalized clothing employing AR somatosensory interactive information technology. This technology makes use of software tools for clothing modelling and the human body. The production process of clothing can save a significant amount of money and time by integrating virtual fitting and stitching. The usage of VR and AR in the production process is discussed in the essay, together with clear equipment and productivity-boosting visualization. In addition, it highlights the use of digital models to expedite the process and notes the fashion industry's sluggish acceptance of 3D virtual prototype technology, despite its continuous use. In the context of the Fourth Economics Edition's digital transition, these studies seek to illuminate the ways in which technologies like additive manufacturing, simulation, and 3D modelling might revolutionize the fashion sector.

Keywords: Virtual Reality, Product development process, Apparel industry, opportunities, challenges, Implementation, Personalized clothing customization, Augmented reality, Apparel Industry, Virtual Reality, New product Development, Augmented Reality, CAD, Virtual reality, Design, Automation, Virtual Clothing, 3D CAD, Prototyping, Collaboration.

Paper type: General review

\section*{INTRODUCTION}
The use of Virtual Reality and Augmented Reality technology into several phases of the design, development, and production processes has caused a revolutionary change in the garment business in recent years. The purpose of this review article is to examine the various ways that VR and AR technologies can be used in the design, development, and manufacturing of clothing. Through a comprehensive examination and synthesis of the current body of literature, our aim is to clarify the developments, obstacles, and potential opportunities linked to the implementation of these state-of-the-art instruments in the fashion sector.

The clothing industry is going through a paradigm change. It used to be a world of static samples and physical designs. Virtual reality (VR) and augmented reality (AR) are two examples of how technological innovations are permeating every aspect of design and production processes. These immersive technologies stimulate extraordinary creativity, productivity, and customer engagement by opening a wealth of possibilities. AR and VR are quickly becoming essential tools for clothing firms looking to stay ahead of the curve in today's intensely competitive market.

But before diving into the exciting applications of these technologies, let's clear up a potential point of confusion: the difference between AR and VR. Virtual Reality (VR) transports users into a completely simulated, computer-generated environment accessed through headsets. Imagine stepping into a virtual fitting room, surrounded by digital projections of garments tailored just for you. Conversely, Augmented Reality (AR) doesn't replace the real world; instead, it overlays digital information onto the physical world. Consider it a kind of magic mirror that lets you view how an item of clothing might appear on your reflection before you ever enter a store. Often, cell phones or specialized AR glasses are used to accomplish this.

Let's now examine how AR and VR are changing the fashion industry's design process. The advent of 3D design software with integrated AR/VR capabilities is posing a challenge to the conventional methods of sketching and depending only on 2D design tools. For designers, this opens a world of possibilities. They may now realize their ideas in three dimensions instead of being limited to flat graphics. This makes it possible to depict the garment more accurately, giving them the information they need to decide on the details, drape, and general form.

Virtual reality (VR) is eliminating uncertainty in the design and fitting process by enabling the creation of virtual fitting rooms. Designers may now quickly test out various fits, textures, and styles in the simulated setting. As an alternative to conventional techniques that depend on tangible prototypes, this enables rapid iterations and modifications, saving time and\\
resources. AR also has the ability to project designs onto actual objects, like models or even customers. This facilitates a more accurate representation of the final fit and appearance of the garment, improving communication and understanding between clients and designers.

AR and VR have a significant impact that goes much beyond design, optimizing production processes from manufacturing to pattern-making. VR has a lot to offer in the area of pattern generation. Imagine a designer using the same intuitive sense as when working with clay to manipulate and shape a virtual garment in a three-dimensional space. As a result, the process is more error-free and efficient than when creating flat patterns the old-fashioned way. AR is essential to production as well, since it guides manufacturing and assembly processes. With the use of these tools, workers can more accurately and efficiently navigate complicated assembly processes by superimposing digital instructions over their actual workspaces. This can greatly increase production speed while also lowering the possibility of errors.

VR has advantages outside of the store. Production line virtual reality simulations are proving to be quite useful for streamlining processes. Businesses can find bottlenecks and inefficiencies early on by building virtual versions of the whole production process. They can practically handle these problems because to their proactive approach, which lowers expenses and improves operational effectiveness.

AR and VR integration in the clothing business is a revolutionary force, not just a passing trend. These tools streamline production procedures and improve consumer interaction while enabling designers to express their ideas in fresh and interesting ways. We may anticipate far more cutting-edge uses for these technologies as they develop and become more widely available, which will completely alter the landscape of clothing design and manufacturing. The possibilities are virtually boundless, ranging from highly efficient production lines led by augmented reality to individualized experiences in virtual fitting rooms. Without a doubt, the advancement and widespread use of AR and VR will have a significant impact on fashion in the future by bringing about a revolutionary change in the apparel industry through the convergence of creativity and efficiency.

\section*{OBJECTIVE}
The purpose of the document "AR and VR Technologies in Apparel Design" is to investigate and assess how AR and VR technologies are used in the apparel sector, with a particular emphasis on manufacturing, \textbf{prototyping, and design processes}.

The purpose of this paper is to examine how \textbf{traditional apparel design and production processes }are being altered by AR and VR technologies. The research aims to illustrate the benefits and difficulties related to the implementation of these technologies in the clothing sector by assessing their influence.

This study explores the ways in which AR and VR technologies are transforming conventional apparel design and production methods. By evaluating their impact, the study seeks to highlight the advantages and challenges associated with the application of these technologies in the apparel industry.

The purpose of the paper is to investigate how \textbf{AR and VR technology might improve clothing industry manufacturing} processes. This involves evaluating how these technologies enhance overall operational efficiency in the clothing industry, optimize production lines, and lower error rates.

Examining the effects of AR and VR technologies on consumer experience and engagement in the garment industry is another goal. The goal of the study is to determine how consumers' impressions and decisions about what to buy are influenced by the usage of virtual fitting rooms, interactive displays, and tailored experiences.

The research aims to provide insights on how AR and VR technologies may be exploited to overcome hurdles and foster innovation in the garment sector by identifying industry challenges and opportunities connected to their adoption.

\section*{METHODOLOGY}
The literature on AR/VR in the apparel industry primarily focuses on two main areas. Firstly, perspectives on AR/VR applications, such as non-contact body measurement and virtual tryon for online shopping, particularly relevant in the context of increased e-commerce and the COVID-19 pandemic. Consumer attitudes towards these technologies, driven by both utilitarian and hedonic motives, are examined, with factors like innovativeness and economic considerations playing roles. Additionally, regular use of AR/VR applications is suggested to boost consumer confidence and strengthen brand relationships. Secondly, there is a focus on using AR/VR in fashion product design, particularly in garment pattern design and 3D modelling for precise clothing design, especially in areas like sportswear and female clothing

\begin{verbatim}
LIMITATIONS
Cost and Technical Constraints: The use of CAD
technology and VR modelling can be expensive and
requires technical skills. Integration with existing
systems and processes can be complex, impacting
performance during migration
Accuracy and Reliability: While CAD technology can
improve design accuracy, limitations in simulating
real-world garment fit and appearance still exist,
eading to discrepancies etween virtual designs
and physical prototypes.
Information Security and Environmental Impact:
Managing large design information raises concern
about information security and environmental
mpact due to the production and disposal of
electronic components.
Dependence on Technology and Compatibility
Issues: Over-reliance on CAD technology can lead to
overuse of electronic technology, and compatibility
issues arise when different CAD software platform
are used, hindering collaboration.
Usability and Reliance on QR Codes: Limited
usability testing and reliance on QR codes for
product identification pose challenges for th
effectiveness of AR tools in apparel manufacturing
\end{verbatim}

where accuracy is crucial. These technologies are vital for tasks like virtual try-on and avatar development, enhancing the design process and manufacturing efficiency in the apparel industry.

\section*{Step 1: Selection of Database for literature search}
The bibliometric analysis was initiated through the identification of the database for carrying out the search for research papers in the domain of AR and VR in apparel. The databases of Research Gate, Google Scholar, Tandfonline, Hindawi, Mdpi, Scribd were selected for the literature search. The Research Gate database includes over 20,000 journals published by leading publishers such Springer Nature, Wiley, Rockefeller University Press and is the largest collection of peer-reviewed research papers. It includes high-quality peer-reviewed research papers published by known publishers in academic domain. Both these databases are used extensively by researchers to conduct bibliometric studies as they facilitate exporting of the research paper data sets into various bibliometric and network mapping tools.

\section*{Step 2: Keyword Identification}
Identifying keywords is crucial in bibliometric analysis to create a thorough list, ensuring a comprehensive search across all relevant research articles within the chosen knowledge domain. articles in the chosen knowledge domain fall into the purview of the search exercise. The search for research literature was carried out using the keyword string "augmented reality" OR "virtual reality" OR "product development process" OR "Personalized clothing customization" OR "AR" OR "VR") AND ("apparel" OR ", CAD" OR "Design", "Automation", "Prototyping".

\section*{Step 3: Initial Results}
The initial search resulted in 12 papers in research gate, Tandfonline 2 Research papers Mdpi 4 Research papers, Google Scholar 8 Research papers, Hindawi 2 Research papers, Scribd 4 Research papers.

\section*{Step 4: Application of Filters}
To narrow down the initial dataset for further analysis, multiple filters are applied to include only relevant research literature. Accordingly Research Gate, Google Scholar, Tandfonline, Hindawi, Mdpi, Scribd have relevant papers of the topics like AR VR Designing, AR VR prototyping and manufacturing. We take the period of 2004-2022. Only articles in English were Selected. Out of 32 papers we only selected 26 papers and minimum citations of 8-12 were taken.

\section*{Step 5: Limitation}
After the selection of paper, we narrow down certain limitations that include cost and technical constraints, Accuracy and Reliability, Information security and environment impact, Dependence on Technology, and compatibility.

\section*{REVIEW OF LITERATURE}
\subsection*{1.1 AUGMENTED REALITY AND VIRTUAL REALITY}
AR is a technique that blends real environment with computer-generated digital information in such a way that user perceive it as one environment (Bonetti, 2018). To enhance the users' experience, several images, textual information, videos or other virtual elements are being added into the real environment that can enable users to view and interact with these virtual products through the usage of their smartphones, tablets or wearable headsets (Olsson, 2013).

On the other hand, VR gives an opportunity to the user to immerse in the computer-generated environment. While using VR-based applications, users use a headset to block the visuals from real life (Wei, 2019). In simple terms, imagine putting on special googles that transport you to a completely different world that feels totally real. You can't see or hear anything from the real world around you because you're so immersed in this digital world. These goggles, called head-mounted displays, make everything seem so real that you forget you're wearing them. It's like stepping into a whole new reality without any distractions from the real world. On the other hand, AR augments the real-world environment by blending the graphical objects using graphic computing and object recognition technologies (Bonetti, 2018). Additionally, AR and VR technologies help in manufacturing/production processes, prototyping, and designing in apparel industry.

\subsection*{1.2 AR AND VR TECHNOLOGIES IN APPAREL DESIGNING}
AR and VR technologies plays a crucial role in garment industry. The contribution of Virtual reality in garment manufacturing, under design and prototyping VR allows designers to create and visualize 3D models of garments in a virtual environment. They can experiment with different fabrics, patterns, and styles, refining their designs before physical production. Designers, pattern makers, and manufacturers can review and discuss designs virtually, reducing the need for physical samples (Soliwal, 2024)


\end{document}